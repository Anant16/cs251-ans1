\documentclass{article}
\usepackage[utf8]{inputenc}
\usepackage{algorithm}
\usepackage{algpseudocode}
\usepackage{amsmath}
\usepackage{graphicx}
\usepackage{hyperref}
\hypersetup{
    colorlinks=true,
    linkcolor=blue,
    filecolor=magenta,      
    urlcolor=cyan,
}
\urlstyle{same}
 
\usepackage[english]{babel}
\usepackage[
backend=biber,
style=alphabetic,
sorting=ynt
]{biblatex}
\addbibresource{wikicite.bib}

\title{CS251 Assignment-7}
\author{Anant Vats }
\date{March 2017}

\begin{document}

\maketitle
\section{About Atharvaveda}
The Atharva (Atharvaveda from atharvānas and veda meaning "knowledge") is the "knowledge storehouse of atharvānas, the procedures for everyday life". The text is the fourth Veda, but has been a late addition to the Vedic scriptures of Hinduism.\cite{wikicite1}

The Atharvaveda is composed in Vedic Sanskrit, and it is a collection of 730 hymns with about 6,000 mantras, divided into 20 books. About a sixth of the Atharvaveda text adapts verses from the Rigveda, and except for Books 15 and 16, the text is in poem form deploying a diversity of Vedic meters.\cite{wikicite1} 

\section{Method of Multiplication}
The following algorithm multiplies two digit numbers quickly.
\begin{algorithm}
\caption{Multiply}\label{Merge}
\begin{algorithmic}[1]
\Procedure{Multiply}{$ab,cd$}\Comment{Multiply special two digit numbers}
\If{$a=c\ \&\ b+d=10$}  \Comment{Case 1}
\State $x\gets a\cdot (a+1)$
\State $y\gets b\cdot d$
\State $Ans\gets xy$
\EndIf
\If{$a+c=10\ \&\ b=d $} \Comment{Case 2}
\State $x\gets a\cdot c+b$
\State$y\gets b\cdot b$
\State$Ans\gets xy$
\Else
\State Do it normally.
\EndIf
\EndProcedure
\Statex

\end{algorithmic}
\end{algorithm}

\section{Is It Worth Remembering}

Effectively, this method requires us to multiply two single-digit numbers, twice. The fastness of the procedures lies in this fact alone. Multiplying single-digit numbers takes hardly any time (we have been doing it since ages). This is for git assignment.

\section{Algebraic Proof}
Consider the two digit numbers\footnote{example a=1 and b=2 is 12}: ab and cd.\\
\Large{\textbf{ Case 1}}
\begin{align}
a=c&\ \& \ b+d=10\\
ab\cdot cd&=(10a+b)\cdot(10a+d)\\
&=100a^2+10a\cdot (b+d)+b\cdot d\\
&=100a^2+10a\cdot (10)+b\cdot d\\
&=100a\cdot(a+1)+b\cdot d
\end{align}
\Large{\textbf{ Example}}\\
\begin{figure}[h]
  \centering
    \includegraphics{ex11.pdf}
    \includegraphics{ex1_11.pdf}
  \caption{Case 1}
  \label{fig:case1}
\end{figure}\\
In Figure \ref{fig:case1}, we see examples of the first type.
\newpage
\Large{\textbf{ Case 2}}
\begin{align}
a+c & =10\ \& \ b=d\\
ab\cdot cd&=(10a+b)\cdot(10c+b)\\
&=100a\cdot c+10b\cdot (a+c)+b^2 \\
&=100a\cdot c+10b\cdot (10)+b^2\\
&=100(a\cdot c+b) + b^2
\end{align}
\Large{\textbf{ Example}}\\
\begin{figure}[h]
  \centering
    \includegraphics{ex21.pdf}
    \includegraphics{ex31.pdf}
  \caption{Case 2}
  \label{fig:case2}
\end{figure}\\
In Figure \ref{fig:case2}, we see examples of the second type.


\section{Summary}
For two digit numbers $ab$ and $cd$. We can multiply as:\\

\begin{tabular}{|c|c|c|}
\hline
    \textbf{Case} & \textbf{Condition} & \textbf{Result}\\
    \hline \hline
     1 & $a=c\ \&\ b+d=10$ & $a\cdot (a+1):b\cdot d$ \\ \hline
     2 & $a+c=10\ \&\ b=d$ & $a\cdot c+b:b^2$ \\
\hline
\end{tabular}


\newpage
\section*{Generalization}

This concept can be extended to numbers of $n$ digits in both cases.\\
Consider two numbers, $N_1,N_2$, of $n$ digits each. Let the digit in one's place be $a,b$ respectively. Let the numbers excuding the one's place be $n_1,n_2$ respectively.\\
Then,\\

\textbf{Case 1:}
\begin{gather}\notag
\begin{aligned}
    n_1 &=n_2\\
a+b &=10\\
    \\
N_1\cdot N_2 &=n_1\cdot (n_1+1):a\cdot b
\end{aligned}
\end{gather}

\textbf{Case 2:}
\begin{gather}\notag
\begin{aligned}
n_1+n_2 &=10^{n-1}\\
    a &=b\\
    \\
N_1\cdot N_2 &=n_1\cdot n_2+a:a^2
\end{aligned}
\end{gather}
The proof for the above cases is similar to the proof for numbers with 2-digits.

\section{Further Reading}
For some more interesting Vedic Math tricks, see:\\
\href{http://www.jainmathemagics.com/vedicmathematics/}{JainMathematics}\\
\href{http://www.mastermindvedicmaths.com/}{Vedic Mastermind}%\\
%\href{http://vedicmathsindia.org/}{Vedic Maths India}


\printbibliography

\end{document}

